\documentclass{book}

\usepackage{amsmath}
\usepackage[latin1]{inputenc}

\usepackage[pdftex,
        colorlinks=true,
        urlcolor=rltblue,       % \href{...}{...} external (URL)
        filecolor=rltgreen,     % \href{...} local file
        linkcolor=rltred,       % \ref{...} and \pageref{...}
        pdftitle={Untitled},
        pdfauthor={b3j0f},
        pdfsubject={D\'emonstration de la conjecture de Syracuse},
        pdfkeywords={syracuse godaltz demonstration conjecture},
        pdfadjustspacing=1,
        pagebackref,
        pdfpagemode=None,
        bookmarksopen=true]{hyperref}

\title{D\'emonstration de la conjecture de Syracuse}
\author{b3j0f}
\date{\today}

\begin{document}\label{start}

\maketitle

\section{Probl\`eme}

La conjecture de syracuse veut que :

Soit la suite d'entiers naturels non nuls $U_n$ tel que:

\[U_{n+1} = \left\{
\begin{array}{l l}
  U_n/2 & \quad \text{si $U_n$ est pair}\\
  3Un+1 & \quad \text{si $U_n$ est impair}\\ \end{array} \right. \]

Alors $\forall U_n \in N$, $\exists m > n$ tel que $U_m = 1$.

\section{D\'emonstration}

D\'emonstration par r\'ecurrence du comportement de $U_n$ dans les sous-ensembles $R$ d'entiers ferm\'es $[2^m; 2^{m+2}[$ o\`u $m$ est pair et $\in N$.

Observons le comportement de $U_n$ sur les ensembles $R0=[1; 4[$, $R1=[4; 16[$ et $R2=[16; 64[$.

Pour observer ce comportement, je ne m'attarderai que sur les valeurs impairs de $U_n$. C'est pourquoi je vais d\'efinir la fonction $F: N* -> N*$ telle que :

\[F(x) = \left\{
\begin{array}{l l}
  F(x/2) & \quad \text{si $x$ est pair}\\
  x & \quad \text{si $x$ est impair}\\ \end{array} \right. \]

Les tableaux suivants comprennent tous les entiers impairs de $R0$, $R1$, et $R2$, avec respectivement la valeur de $F$, un indicatif de s\'erie \b{remarquable} que j'expliquerai par la suite.

\begin{tabular}{|c|c|c|}
	\hline
	$R0$ & $F$ & Serie \\
	\hline
	1 & 1 & 2 \\
	\hline
	3 & 5 & 1 \\
	\hline
\end{tabular}

\begin{tabular}{|c|c|c|}
	\hline
	$R1$ & $F$ & Serie \\
	\hline
	5 & 1 & 3 \\
	\hline
	7 & 11 & 1 \\
	\hline
	9 & 7 & 2 \\
	\hline
	11 & 17 & 1 \\
	\hline
	13 & 5 & 3 \\
	\hline
	15 & 23 & 1 \\
	\hline
\end{tabular}

\begin{tabular}{|c|c|c|}
	\hline
	$R2$ & $F$ & Serie \\
	\hline
	17 & 13 & 2 \\
	\hline
	19 & 29 & 1 \\
	\hline
	21 & 1 & 3 \\
	\hline
	23 & 35 & 1 \\
	\hline
	25 & 19 & 2 \\
	\hline
	27 & 41 & 1 \\
	\hline
	29 & 11 & 3 \\
	\hline
	31 & 47 & 1 \\
	\hline
	33 & 25 & 2 \\
	\hline
	35 & 53 & 1 \\
	\hline
	37 & 7 & 3 \\
	\hline
	39 & 59 & 1 \\
	\hline
	41 & 31 & 2 \\
	\hline
	43 & 65 & 1 \\
	\hline
	45 & 17 & 3 \\
	\hline
	47 & 71 & 1 \\
	\hline
	49 & 37 & 2 \\
	\hline
	51 & 77 & 1 \\
	\hline
	53 & 5 & 3 \\
	\hline
	55 & 83 & 1 \\
	\hline
	57 & 43 & 2 \\
	\hline
	59 & 89 & 1 \\
	\hline
	61 & 23 & 3 \\
	\hline
	63 & 95 & 1 \\
	\hline
\end{tabular}

\subsection{D\'ecoupage de $R$ en suites remarquables}

Afin de mieux isoler des comportements r\'ecursifs de la suite $U_n$, je vais transformer l'ensemble des entiers naturels en trois suites bien distinctes (sans collision).

\subsubsection{Suite 1 : $S_{1,n}$}

Premi\`erement, int\'eressons-nous \`a la suite 1.

Cette suite a plusieurs propri\'et\'es remarquables :

\paragraph{Valeurs}

Ses valeurs correspondent \`a la fonction lin\'eaire $y = 3 + 4x$, $\forall x \in N$.

Par exemple :

\begin{itemize}
	\item $3 = 4 + 4 * 0$
	\item $7 = 3 + 4 * 1$
	\item $11 = 3 + 4 * 2$
	\item etc.
\end{itemize}

\paragraph{Application de $F$}

L'application de $F$ est donn\'ee par la fonction lin\'eaire $y = 5 + 6x$, $\forall x \in N$.

Par exemple :

\begin{itemize}
	\item $F(3) = 5 = 5 + 6 * 0$
	\item $F(7) = 11 = 5 + 6 * 1$
	\item $F(11) = 17 = 5 + 6 * 2$
	\item etc.
\end{itemize}

Et pour tout $S_{1,n}$, on a $F_n(S_{1,n}) = S_{1,n} * 2 - 2 * n - 1$.

\subsubsection{Serie 2 : $S_{2,n}$}

La seconde suite est tr\`es proche de la premi\`ere.

\paragraph{Valeurs}

Les valeurs de cette suite sont d\'etermin\'ees par la fonction lin\'eaire $y = 1 + 8x$, $\forall x \in N$.

Par exemple :

\begin{itemize}
	\item $1 = 1 + 8 * 0$
	\item $9 = 1 + 8 * 1$
	\item $17 = 1 + 8 * 2$
\end{itemize}

\paragraph{Application de $F$}

L'application de $F$ correspond \`a la fonction lin\'eaire $y = 1 + 6x$, $\forall x \in N$.

Par exemple :

\begin{itemize}
	\item $F(1) = 1 = 1 + 6 * 0$
	\item $F(9) = 7 = 1 + 6 * 1$
	\item $F(17) = 13 = 1 + 6 * 2$
\end{itemize}

Et pour tout \'el\'ement de $S_2$, on a $F(S_{2,n}) = S_{2,n} * 2  - 2 * n$.

\subsubsection{Suite 3 : $S_{3,n}$}

Cette derni\`ere suite est plus particuli\`ere.

\paragraph{Valeurs}

Les valeurs de $S_{3,n}$ sont d\'etermin\'ees par la fonction lin\'eaire $y = 5 + 8n$, $\forall n \in N$.

Par exemple :

\begin{itemize}
	\item $5 = 5 + 8 * 0$
	\item $13 = 5 + 8 * 1$
	\item $21 = 5 + 8 * 2$
	\item $29 = 5 + 8 * 3$
	\item $37 = 5 + 8 * 4$
	\item $45 = 5 + 8 * 5$
	\item $53 = 5 + 8 * 6$
	\item $61 = 5 + 8 * 7$
\end{itemize}

\paragraph{Application de $F$}

Contrairement aux pr\'ec\'edentes suites, cette fois, l'application de $F$ n'est pas lin\'eaire mais successivement croissante et d\'ecroissante.

Par ailleurs, l'application de $F$ provient directement des applications de $F$ dans l'ensemble $R_{x-1}$, $x \in N$ qui pr\'ec\'ede celui de $S_{3,n} \in R_x$.

On a donc : $F(S_{3,n}) = F(\frac{S_{3,n} - 1}{4})$

Par exemple :

\begin{itemize}
	\item $F(21) = F(\frac{21 - 1}{4}) = F(5) = 1$
	\item $F(29) = F(\frac{29 - 1}{4}) = F(7) = 11$
	\item $F(37) = F(\frac{37 - 1}{4}) = F(9) = 7$
	\item $F(45) = F(\frac{45 - 1}{4}) = F(11) = 17$
	\item $F(53) = F(\frac{53 - 1}{4}) = F(13) = 5$
	\item $F(61) = F(\frac{61 - 1}{4}) = F(15) = 23$
\end{itemize}

\subsection{Conclusion}

Par l'observation, nous avons montr\'e un d\'ecoupage de l'ensemble $N$ en sous-ensembles $R_x$, $x \in N$.
Dans ces sous-ensembles, nous avons observ\'e des suites disjointes qui couvrent l'ensemble des entiers non nuls en appliquant la fonction $F$.

De plus, nous avons un cycle comportemental de ces suites tel que :

Les suites $S_{1,n}, S_{2,n} et S_{3,n}$ se suivent successivement et recursivement dans cet ordre :

$\{S_{2,n}, S_{1,n}, S_{3,n}, S_{1,n+1}\}$.

De plus, nous avons un respect des proportions o\`u $\forall x \in N$, $|R_x| * 2 = |R_{x+1}|$

En sachant que pour $R_0$, toutes les valeurs de cet ensemble valident la conjecture de Syracuse et que $R_1$ est valid\'e par proportionalit\'e des comportements alors par r\'ecurrence, la conjecture de Syracuse est vrai quelque soit l'ensemble $R_x$, $\forall x \in N$ et donc par transivit\'e sur $N*$.

\label{end}\end{document}
